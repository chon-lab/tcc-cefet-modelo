\chapter{Introdução}
\label{cap:introducao}
A contextualização é a primeira parte da introdução e tem como objetivo situar o leitor sobre o tema do trabalho. É aqui que o aluno deve apresentar o cenário geral da área abordada, explicando os principais conceitos, tendências, desafios ou avanços relacionados ao tema escolhido.

Deve-se começar com uma visão ampla, destacando aspectos relevantes da área de estudo e, gradualmente, direcionar o texto até chegar ao foco específico do trabalho. O aluno pode mencionar dados estatísticos, referências teóricas, aplicações práticas ou acontecimentos recentes que justifiquem a importância do tema. A ideia é responder à pergunta: por que este tema é relevante hoje?

Além disso, é importante destacar se o tema é atual, inovador, pouco explorado ou se existe alguma lacuna ou demanda prática que ainda não foi suficientemente atendida. Sempre que possível, a contextualização deve ser fundamentada com referências bibliográficas confiáveis, preferencialmente recentes e pertinentes à área.

Essa parte deve gerar no leitor interesse pelo trabalho e preparar o terreno para a apresentação do problema de pesquisa na seção seguinte.

\section{Problema} 
Nesta seção, o aluno deve descrever o problema central que motivou o desenvolvimento do trabalho. Isso envolve contextualizar a situação, mostrar lacunas existentes na área ou desafios enfrentados, e justificar a importância de tratar essa problemática. A redação deve ser objetiva, baseada em fatos e, sempre que possível, sustentada por referências bibliográficas.

Aqui devem ser formuladas uma ou mais perguntas que o trabalho busca responder. As questões de pesquisa ajudam a delimitar o foco do estudo e orientam a construção dos objetivos. Elas devem ser claras, específicas e relacionadas ao problema descrito anteriormente.

\begin{quote}
\textit{QP 1: Colocar as questões de pesquisa de forma clara e direta?}
\end{quote}

\section{Objetivo}
Esta seção apresenta o objetivo geral do trabalho, ou seja, a finalidade principal que se pretende alcançar. O objetivo deve estar diretamente ligado ao problema identificado e às questões de pesquisa. A redação deve começar com verbos no infinitivo, como desenvolver, analisar, avaliar, propor, entre outros.

\subsection{Objetivos Específicos}
Os objetivos específicos detalham as etapas intermediárias ou metas que serão cumpridas para atingir o objetivo geral. Cada objetivo específico deve ser claro, mensurável e realizável dentro do escopo do trabalho.

\begin{enumerate}
    \item \textbf{O primeiro objetivo} e uma breve descrição.
\end{enumerate}

\section{Metodologia}
Neste item, o aluno deve descrever o caminho metodológico adotado no trabalho. Isso inclui os métodos de pesquisa (qualitativa, quantitativa, experimental etc.), técnicas utilizadas (levantamento bibliográfico, estudo de caso, modelagem, implementação, testes etc.), ferramentas empregadas (softwares, linguagens, frameworks), e o delineamento geral do processo. O nível de detalhamento deve ser suficiente para que outro pesquisador possa compreender e, eventualmente, replicar o estudo.

\section{Contribuições}
Nesta seção, o aluno deve explicitar quais são as principais contribuições do trabalho. Isso pode incluir avanços teóricos, práticos ou tecnológicos; o desenvolvimento de uma ferramenta; uma nova abordagem metodológica; ou a solução de um problema específico. A ênfase deve ser no valor agregado pelo trabalho dentro de seu contexto acadêmico ou científico.

\section{Estrutura do trabalho}
Aqui se apresenta a organização dos capítulos do TCC. Deve-se descrever o que é abordado em cada capítulo, dando ao leitor uma visão geral do conteúdo e da lógica do documento