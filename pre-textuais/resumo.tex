\dresumo{
O resumo é uma parte essencial do TCC, pois apresenta de forma concisa os principais pontos do trabalho. Deve ser escrito em um único parágrafo, com aproximadamente 150 a 500 palavras, e deve conter as seguintes informações: Primeiramente, é importante apresentar o tema e o contexto do trabalho, explicando brevemente em que área ele se insere e qual problema ou motivação o justifica. Em seguida, o aluno deve deixar claro qual é o objetivo principal da pesquisa ou do projeto desenvolvido. Depois, deve-se descrever, de forma sucinta, a metodologia adotada — ou seja, quais foram os métodos, ferramentas, técnicas ou etapas utilizadas para atingir o objetivo proposto. O resumo também deve apresentar os principais resultados obtidos, ainda que sejam parciais, destacando as evidências mais relevantes. Por fim, deve-se incluir a conclusão principal do trabalho, ressaltando sua contribuição ou impacto. O resumo deve ser informativo, escrito em linguagem clara e objetiva, sem o uso de citações, siglas não definidas ou fórmulas matemáticas. Ele deve ser compreensível por qualquer leitor, mesmo que não tenha acesso ao restante do trabalho. Ao final do resumo, o aluno deve indicar cinco palavras-chave, separadas por ponto e vírgula, que representem os principais temas abordados no TCC.
}
{Palavra-chave 1; Palavra-chave 2; Palavra-chave 3; Palavra-chave 4; Palavra-chave 5}
